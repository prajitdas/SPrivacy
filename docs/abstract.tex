\begin{abstract}
Some abstract
\hl{R: review. The structure should be: 1) motivation, 2) what has been done before, 3) goal (context enrichment), 4) method (ontology, reasoner, sharing, inconsistency checking)}
\end{abstract}

% IEEEtran.cls defaults to using nonbold math in the Abstract.
% This preserves the distinction between vectors and scalars. However,
% if the conference you are submitting to favors bold math in the abstract,
% then you can use LaTeX's standard command \boldmath at the very start
% of the abstract to achieve this. Many IEEE journals/conferences frown on
% math in the abstract anyway.

% \begin{keywords)
% Mobile Applications, Access controls.
% \end{keywords}

% For peer review papers, you can put extra information on the cover
% page as needed:
% \ifCLASSOPTIONpeerreview
% \begin{center} \bfseries EDICS Category: 3-BBND \end{center}
% \fi
%
% For peerreview papers, this IEEEtran command inserts a page break and
% creates the second title. It will be ignored for other modes.
%c\IEEEpeerreviewmaketitle